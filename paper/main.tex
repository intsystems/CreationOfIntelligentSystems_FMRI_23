\documentclass{article}

\usepackage{arxiv}

\usepackage[utf8]{inputenc} % allow utf-8 input
\usepackage[T1]{fontenc}    % use 8-bit T1 fonts
\usepackage{hyperref}       % hyperlinks
\usepackage{url}            % simple URL typesetting
\usepackage{booktabs}       % professional-quality tables
\usepackage{amsfonts}       % blackboard math symbols
\usepackage{nicefrac}       % compact symbols for 1/2, etc.
\usepackage{microtype}      % microtypography
\usepackage{cleveref}       % smart cross-referencing
\usepackage{lipsum}         % Can be removed after putting your text content
\usepackage{graphicx}
\usepackage[english, russian]{babel}
\usepackage{natbib}
\usepackage{doi}
\usepackage{subcaption}

\title{FITS: fMRI temporal and sparse relations}

% Here you can change the date presented in the paper title
%\date{September 9, 1985}
% Or remove it
%\date{}

\newif\ifuniqueAffiliation
% Uncomment to use multiple affiliations variant of author block 
\uniqueAffiliationtrue

\ifuniqueAffiliation % Standard variant of author block
\author{Galina Boeva \\
	MIPT\\
	Antiplagiat\\
	  Skoltech \\
	%% examples of more authors
 \and Nikita Kornilov \\
 MIPT \\
 Skoltech
 \and Marat Khusainov \\
 MIPT 
 \and Bair Mihailov \\
 MIPT\\
 Skoltech
}
\else
% Multiple affiliations variant of author block
\usepackage{authblk}
\renewcommand\Authfont{\bfseries}
\setlength{\affilsep}{0em}
% box is needed for correct spacing with authblk
\newbox{\orcid}\sbox{\orcid}{\includegraphics[scale=0.06]{orcid.pdf}} 
\author[1]{%
	\href{https://orcid.org/0000-0000-0000-0000}{\usebox{\orcid}\hspace{1mm}David S.~Hippocampus\thanks{\texttt{hippo@cs.cranberry-lemon.edu}}}%
}
\author[1,2]{%
	\href{https://orcid.org/0000-0000-0000-0000}{\usebox{\orcid}\hspace{1mm}Elias D.~Striatum\thanks{\texttt{stariate@ee.mount-sheikh.edu}}}%
}
\affil[1]{}
\affil[2]{}
\fi

% Uncomment to override  the `A preprint' in the header
%\renewcommand{\headeright}{Technical Report}
%\renewcommand{\undertitle}{Technical Report}
\renewcommand{\shorttitle}{\textit{arXiv} Template}

%%% Add PDF metadata to help others organize their library
%%% Once the PDF is generated, you can check the metadata with
%%% $ pdfinfo template.pdf
\hypersetup{
pdftitle={A template for the arxiv style},
pdfsubject={q-bio.NC, q-bio.QM},
pdfauthor={David S.~Hippocampus, Elias D.~Striatum},
pdfkeywords={First keyword, Second keyword, More},
}
\renewcommand{\vec}[1]{\mathbf{#1}}
\newcommand{\R}[]{\mathbb{R}}
\begin{document}
\maketitle

\begin{abstract}
В данном исследовании предлагается рассмотреть пространственно-временную модель визуализации временного ряда данных fMRI (Функциональная магнитно-резонансная томография) для предсказания на основе целевого фрагмента fMRI видео. В качестве решения мы предлагаем новую технику глубокого обучения основанную на механизме внимания, примененного как и к пространственной составляющей данных, так и к временной компоненте. Итоговая модель обладает высокой скоростью  работы и высокими метриками качества предсказаний.      
\end{abstract}


% keywords can be removed
\keywords{First keyword \and Second keyword \and More}


\section{Введение}

В данном исследовании предлагается рассмотреть пространственно-временную модель визуализации временного ряда данных fMRI (Функциональная магнитно-резонансная томография) для предсказания на основе целевого фрагмента fMRI видео. При просмотре видеороликов в интернете, фильмов, Tik-tok, наш мозг реагирует, что приводит в возбуждение разные отделы мозга. Например, средний мозг реагирует на зрительные раздражения, в свою очередь задний отдел реагирует на мимику человека. fMRI позволяет определить активность мозга в тот или иной момент времени, так как при влиянии раздражителей мы можем выявить зависимость реакции конкретной части головного мозга.

В данной области основной проблемой является решение трудностей с огромной размерностью данных. В ряде статей~\cite{ rezaei2023deep, rezaei2020bayesian} предлагается применение модели декодера для  многомерных наблюдений, используя теорему Байеса и условие марковости для входных данных. Также использовался байесовских подход и в статье ~\cite{yousefi2019decoding}, авторы изучают когнитивные процессы, которые в свою очередь тоже являются многомерным временным рядом. Предлагается использовать структуру моделирования, в которой когнитивный процесс определяется как низкоразмерная динамическая латентная переменная, называемая когнитивным состоянием. Для моделирования всех этих взаимосвязей применяется архитектура энкодер-декодер. 

Сопутствующей к большой размерности данных  проблемой является высокое время обработки видео и fMRI. В работе \cite{rezaei2021real} авторы предложили вместо схемы энкодер-декодер использовать смесь гауссиан для моделирования отклика каждой клетки fMRI, тем самым сократив время вычислений до приемлемого на практике. Для обучения использовался принцип максимизации вариационной нижней оценки ELBO. Однако в статье рассматривались только низкоразмерные данные без пространственной составляющей, что является плохим приближением fMRI. 

Анализ данных fMRI проводился многими авторами. Интересным подходом является применение методов, которые могут учитывать как пространственную, так и временную компоненту в данных. Временная компонента это распределение каждого отдельного признака, отвечающего за конкретный отдел мозга, по временной шкале. Пространственная в свою очередь отвечает за связь различных признаков определенных отделов мозга друг с другом, которые могут описывать разные моменты времени. Например, в статье ~\cite{azevedo2022deep} авторами рассматривается метод, основанный на двух блоках. Первый - это сверточная временная сетка для работы с временной компонентой данных. Вторая - это граф для связи признаков, которые распределены в пространстве, то есть учет пространственной компоненты.  
В работе ~\cite{malkiel2022selfsupervised} используется трансформер, обученный в 2 этапа: сначала заполнение пропусков в видео fMRI, затем предсказание целевой величины. 
Другим примером анализа данных fMRI является работа ~\cite{Albuquerque_FMRI}. Авторами предложен гибридный подход, объединяющий глубокие порождающие модели и гауссовские процессы в обобщенной аддитивной модели.

Так как использования гауссовских процессов в моделях вариационных автокодировщиков может играть одну из ключевых ролей в нашей задаче, стоит отметить работу ~\cite{dai2016variational}, которая по сути является пионерской в данной области. Авторами была предложена модель, которая состоит из нескольких слоев скрытых переменных и использует гауссовские процессы для отображения между последовательными слоями. Несмотря на то, что в статье рассматривается реализация данной модели на низкоразмерных датасетах, она послужила фундаментом для развития данной области исследования. Гауссовские процессы нашли свое применение не только в моделях вариационных автокодировщиков. Так, в работах ~\cite{jankowiak2020parametric} и ~\cite{wang2022intuitive}рассматривается их применение в задаче регрессии. В случае регрессии часто прогнозируемая дисперсия преобладает над шумом во входных данных. Кроме того,  оценки неопределенности,  практически не используют неопределенность функции, зависящей от входных данных. Обе работы решают эту проблему и, таким образом, дают существенно улучшенные оценки неопределенности прогнозирования.


В работе ~\cite{matsubara2019deep} рассматриваются подходы, которые помогают работать с сигналами, не интересующие нас при предсказании. Так как данные многомерны, то и признаком порождается в много раз больше, что позволяет утверждать, что нужно проводить фильтрацию. Предложена генеративная модель на данных fMRI в состоянии покоя. Предлагаемая модель обусловлена предположением о состоянии субъекта и оценивает апостериорную вероятность состояния субъекта с учетом данных визуализации, используя теорему Байеса.

Генерация fMRI по видео является востребованной задачей. В частности, в 2021 году соревнование The Algonauts project было посвящено именно этой задаче ~\cite{DBLP:journals/corr/abs-2104-13714}. В работе ~\cite{Yang2021.08.24.457581}победителей использовался ансамбль моделей, каждая из которых фокусировалась на различных областях видео: движение, границы объектов, звук и т.п.


\section{Мaтематическая постановка}
\subsection{Формулировка}
В этой главе мы представляем математическую формулировку нашей задачи. Пусть дан временной ряд из $T$ элементов fMRI и видео, а именно
$$\{(\vec{X}_t, \vec{V}_t)\}_{t= 1}^T,$$
где переменная $\vec{X}_t \in \R^{H \times W\times D}$ отвечает за fMRI снимок размера $H \times W\times D$ в момент времени $t$, а переменная $\vec{V}_t \in \R^{H_v \times W_v \times C}$ за кадр видео размера $H_v \times W_v$ с $C$ каналами. Будем считать, что видеоряд и fMRI являются согласованными, то есть реакции $\vec{X}_t$ происходит на фрагмент видео $\vec{V}_t$ в тот же самый момент времени.

 Задача заключается в прогнозировании fMRI ряда на $N$ моментов времени наперёд. Для этого мы будем обучать модель $M_\theta$ с параметрами $\theta$, которая будет по входу $ \{\vec{X}_t\}_{t=1}^{K-1}, \{ \vec{V}_t\}_{t= 1}^{K}$  произвольной длины $K$ предсказывать следующий снимок $\vec{X}_K$. Также введём функцию различия двух кадров fMRI $\mathcal{L}: \R^{H \times W\times D} \times  \R^{H \times W\times D} \rightarrow \R$, и будем минимизировать среднюю ошибку по всем доступным моментам времени 

 \begin{equation}\label{eq:min_task}
     \min\limits_{\theta} \frac1T \sum_{k=1}^T  \mathcal{L}(\vec{X}_k, M_\theta(\{\vec{X}_t\}_{t=1}^{k-1}, \{ \vec{V}_t\}_{t= 1}^{k})).
 \end{equation}

 Оптимизацию этого функционала мы будем проводить градиентными методами первого порядка по типу Adam \cite{kingma2014adam}, RMSProp.

\subsection{Описание данных}

Как мы упоминали раньше данные FMRI, представляют из $4$-мерные тензоры, в которых $3$ размерности пространственные и $1$ размерность - временная. На рисунках ниже представлены разные размерности одного конкретного сэмпла из обучающей выборки. Цвет пикселя соответствует интенсивности отклика мозга.

\begin{figure}[h!]
        \centering
        \includegraphics[scale=0.3]{images/сиз1.png} 
        \caption{Срез головного мозга}
        \label{fig:enter-label}
\end{figure}

\begin{figure}[h!]
        \centering
        \includegraphics[scale=0.5]{images/сиз2.png} 
        \caption{Пространственная компонента}
        \label{fig:enter-label}
\end{figure}

\begin{figure}[h!]
        \centering
        \includegraphics[scale=0.5]{images/fmri_timestamp1.png} 
        \caption{Временная компонента при фиксированном горизонтальном срезе ($W = const$) головного мозга}
        \label{fmri_timestamp1}
\end{figure}
\newpage
\begin{figure}[h!]
        \centering
        \includegraphics[scale=0.5]{images/fmri_timestamp2.png} 
        \caption{Временная компонента при фиксированном вертикальном срезе ($H = const$) головного мозга}
        \label{fmri_timestamp2}
\end{figure}

В данной работе мы предполагаем следующую гипотезу относительно наших данных: fMRI снимки, генерируемые реальным человеком, обладают свойством марковости, т.е.
$$P(\vec{X}_t | \vec{X}_{t-1}, \vec{X}_{t-2}, \dots, \vec{X}_1 ) = P(\vec{X}_t| \vec{X}_{t-1}).$$

Такую гипотезу мы вводим ввиду того, что в процессе просмотра видеоролика, fMRI изменяется крайне мало под действием конкретных кадров. Таким образом, в самой модели мы можем обрабатывать не полный ряд $\{\vec{X}_t\}_{t=1}^T$, а лишь два  последовательных кадра $\vec{X}_{T}, \vec{X}_{T-1}$.

\subsection{Описание модели}
% В этой главе мы рассмотрим то множество моделей $M_\theta$, из которого будем искать оптимальные гиперпараметры и параметры.  Для того чтобы учесть и временную, и пространственную зависимости, мы воспользуемся принципами работы механизма внимания, предложенного для fMRI в работе \cite{malkiel2022selfsupervised}. В частности, для перевода fMRI и видео будут использоваться  инкодеры 3DEncoder и 2DEncoder соответственно, после этого выводы инкодеров конкатенируют и отправляются в декодер блок трансформера TRANS, предсказывающий следующий снимок fMRI в скрытом пространстве. После с помощью 3dDecoder происходит преобразование скрытого представления в полноценный fMRI снимок.  

% Модель состоит из TFF для обработки входного кадра и энкодера для обработки видео, который преобразует его в вектор признаков. Последовательность fMRI подается на вход в TFF. Энкодер генерирует среднее значение и вектор дисперсии измерения $emb\_dim$ для каждого измерения. Декодер восстанавливает сигнал из выборки из гауссовского распределения с этими параметрами. Видео энкодер предсказывает векторы среднего значения и дисперсии одного и того же измерения $emb \_dim$ для каждого измерения fMRI из видео-последовательности. Они суммируются со средним значением и дисперсией дополнительного кадра fMRI. Полученный результат используется в качестве априорного распределения для VAE.
В данной главе будет рассмотрена архитектура модели используемой в работе. Ее схема представлена на Рис. \ref{fig:scheme}. Основной составляющей модели является нейросетевая архитектура TFF для работы с данными fMRI (верхний блок), которая была предложена в работе \cite{malkiel2022selfsupervised}.

Модель состоит из трансформера, который работает с векторами признаков, извлеченных энкодером с использованием трехмерных сверток. На первом этапе происходит предобработка данных fMRI. А именно, происходит глобальная и воксельная нормализация кадров видео.  После этого, энкодер оперирует отдельно с трехмерным кадром fMRI, отображая каждый кадр видео последовательности в вектор. Далее, модель агрегирует вектора последовательных кадров в единую последовательность, и она подается в трансформер. Затем, выход трансформера подается в декодер, который восстанавливает оригинальные данные. Для контроля качества восстановленного сигнала используются трехкомпонентная функция потерь:
$$\mathcal{L}_{rec} = \mathcal{L}_1 + \mathcal{L}_1^b + \mathcal{L}_p $$,
где $\mathcal{L}_1$ и $\mathcal{L}_1^b$ являются функциями потерь средней абсолютной ошибки между значениями пикселей восстановленного изображения и оригинального изображения после глобальной и воксельной нормализаций соответственно;
$\mathcal{L}_p $ так же представляет собой функцию потерь средней абсолютной ошибки, однако на признаковом пространстве между сгенерированным кадром и оригинальным.  

% Схема нейронной сети представлена на Рисунке \ref{fig:scheme}.
\begin{figure}[h]
    \centering
    \includegraphics[width=0.6\textwidth]{images/model3.png}
    \caption{Схема модели.}
    \label{fig:scheme}
\end{figure}

% В качестве модели для обработки внешнего видео (нижние блоки) использовалась архитектура VTN, описанная в работе \cite{neimark2021video}.   

\section{Эксперименты}
График обучения модели и результаты её работы представлены на рисунках \ref{pic:training} и \ref{pic:slices} соответственно. Основными метриками качества являются reconstrucion loss, средняя норма отклонения от истинного ответа, и intensity. О них и поговорим подробнее ниже.   
\begin{figure}[h!]
      \centering
      
        \begin{subfigure}[b]{0.4\linewidth}
        \includegraphics[scale=0.4]{images/loss 6 epoch.png} 
        \caption{Функция потерь от восстановления}
       \label{pic:training}
        \end{subfigure}
        \begin{subfigure}[b]{0.4\linewidth}
        \includegraphics[scale=0.4]{images/full loss 6.png} 
        \caption{Полная функция потерь}
        
        \end{subfigure}
        \caption{Процесс обучения модели}
        \label{loss}
        
\end{figure}
\begin{figure}[h!]
      \centering
      
        \begin{subfigure}[b]{0.4\linewidth}
        \includegraphics[scale=0.4]{images/slices 6.png} 
        \label{pic:slices}
        
        \end{subfigure}
        \begin{subfigure}[b]{0.4\linewidth}
        \includegraphics[scale=0.4]{images/slices_6_2.jpg} 
      
        
        \end{subfigure}
        \caption{Результаты работы модели}
        \label{res}
\end{figure}

Начнем с общей метрики оценки качества нашего подхода, мы ссылаемся на статью\cite{malkiel2022self}.

В данной работе используется трехкомпонентная функция потерь, отвечающая за реконструкцию:
\[L_{reconstruction} = L_1 + L_{perceptual} + Loss_{intensity}\]

$L_{perceptual}$ отвечает за минимизацию расстояния $L_2$
между картами признаков первого и второго слоев, извлеченными из восстановленных данных и входных кадров. $L_1$ это стандартная функция потерь, применяемая между выходом декодера и глобально нормализованными кадрами $X_n = \frac{X - \mu}{\sigma}$, это $\mu$ - среднее и $\sigma$ - дисперсия для исходного потока кадров X. $Loss_{itensity}$ - эта потеря основана на $L_1$, применяемом к подмножеству вокселов, связанных с локальными значениями интенсивности, которые, скорее
всего, представляют соответствующий сигнал. Более конкретно, учитывая полное сканирование $(x_1, \dots , x_n)$, мы
выводим нормализацию вокселя $X^v_n$, мы считаем среднее и дисперсию для каждого вокселя и нормализуем его. Затем для каждого нормализованного по вокселу кадра $x^v_i \in X^v$, мы
устанавливаем $x^v_i = 0$, если $|\hat{x}^v_i| < b$ и $x^v_i = \hat{x}^v_i$ в противном случае, где b - настроенное пороговое значение 80\%-ный квантиль абсолютных значений, нормализованных по вокселу. Мотивация, стоящая за устранением вокселов, связанных с 80\% значений, которые ближе к 0, заключается в том, что эти значения типичны для многих кадров и, следовательно, вряд ли будут представлять собой отличительный сигнал. 

На рисунке\ref{res} изображены реконструированные изображения. Что можно заметить, что достаточно много шума. Но в свою очередь область выявлена верно, но с большим разбросом значений. В будущем стоит использовать дополнительную нормализацию для изображений, и регуляризацию при обучении. На рисунке\ref{loss} изображены функции потерь модели модели, наблюдаем, что модель достаточно быстро сходится, то есть обучение происходит корректно и переобучение отсутствует. Конечно, результаты далеки от истинных, но работа достаточно нетривиальная и тут большая область исследований.

\section{Выводы}

В данной работе произведено исследование по анализу видео ряда fMRI данных. Используя подход, основанный на методе TFF, мы обучили модель и получили семплы fMRI снимков в определенные моменты времени, зависящие от истории. Область высокой интенсивности мы определяем верно, но с достаточно большой дисперсией. У нас есть несколько идей по работе с данными недостатками, как, например, регуляризация или нормализация. Также основной идеей в будущем стала применение модели SWIN, которая будет получать вектор отдельного кадра, выявляя тем самым дополнительную взаимосвязь признаков в картинке. Что в будущем, может помочь повысить результат полученных семплов.
\bibliographystyle{unsrtnat}
\bibliography{references} 
\end{document}